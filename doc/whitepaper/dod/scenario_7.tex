% ******************
%  Copyright 2024 TekFive, Inc. and contributors
%
%   Licensed under the Apache License, Version 2.0 (the "License");
%   you may not use this file except in compliance with the License.
%   You may obtain a copy of the License at
%
%     http://www.apache.org/licenses/LICENSE-2.0
%
%   Unless required by applicable law or agreed to in writing, software
%   distributed under the License is distributed on an "AS IS" BASIS,
%   WITHOUT WARRANTIES OR CONDITIONS OF ANY KIND, either express or implied.
%   See the License for the specific language governing permissions and
%   limitations under the License.
% ******************

Despite the odds, the squad has caught up with the target.
Unfortunately, the swarm vanguard has been spotted and enemy countermeasures are dropping micro-drones out of the sky.
A squad specialist requests that the swarms reform into two networks: one to distract the countermeasures, and the other, composed of only three drones, to acquire target triangulation.
The target is positively identified, but is (predictably) in a hardened position.
The squad cannot complete the mission directly.
Fortunately, Command scrambled a fighter jet immediately when the MQ-800 went rogue.
It arrives just in time, and announces itself, validates with the squad/swarm network, and acquires the targeting data all in less than one second.
The pilot fires and receives swarm-drone confirmation of success.
All this within the six seconds of a single, high-speed, low-level pass of the jet.
With just two drones left, the squad high-tails it to the exfiltration point -- one drone looking ahead and the other watching their six.
The mission is successful with a major loss of electronics, but no casualties.
They never even saw the enemy face-to-face.
