a number of independent sites which are requested to work together with an over-arching organization --- say, the Department of Veterans Affairs (VA) spinning up a new service that allows veterans to coordinate VA and non-VA health services coherently.
Such a system would allow non-VA medical practitioners to coordinate care, passing and receiving a veteran's sensitive medical records to and from the VA system using the veteran as a gateway.
Although veterans currently have access to their records, such coordination is manual and requires the veteran's active management of paperwork.

There are numerous roadblocks to implementing such a system today: legal, logistical, administrative, security, and power dynamics.
The Health Insurance Portability and Accountability Act (HIPAA) places strong legal restrictions on exposing medical records to third parties, and many states grant ownership of these records to the hospital or physician.
Secondly, the VA is early in the process of standardizing their electronic records, so there may be inconsistencies among them.
Medical management software between the participants is almost certain to be incompatible and so must be bridged.
Patients frequently do not own their records, so a traceable chain of custody is required to maintain attribution and responsibility in mixed records, and motivate proper secure handling.
Finally, although veteran-patients are strongly driven to find the best care, human social tools make navigating this many-to-one-to-many relationship exceedingly difficult.
Our goal is to bridge competing bureaucracies, which have massive legal and de-facto authority over the patient, thereby empowering the veteran without hobbling practitioners.
