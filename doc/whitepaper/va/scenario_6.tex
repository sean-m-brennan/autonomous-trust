% ******************
%  Copyright 2024 TekFive, Inc. and contributors
%
%   Licensed under the Apache License, Version 2.0 (the "License");
%   you may not use this file except in compliance with the License.
%   You may obtain a copy of the License at
%
%     http://www.apache.org/licenses/LICENSE-2.0
%
%   Unless required by applicable law or agreed to in writing, software
%   distributed under the License is distributed on an "AS IS" BASIS,
%   WITHOUT WARRANTIES OR CONDITIONS OF ANY KIND, either express or implied.
%   See the License for the specific language governing permissions and
%   limitations under the License.
% ******************

Veterans, upon visiting a non-VA specialist for the first time, can scan a QR code that registers their \projectName VA agent with the medical office's servers via local WiFi. On this initial visit, the menu of data services available to choose from is limited and quite general.
However as the veteran's relationship with this office grows, so too does the list of services and the ability to coordinate care when visiting the local VA grow.
It may start with mere chart-sharing then expand to intake comparisons, prescription and pharmacy coordination, and so on.
The veteran simple selects certain services and the details are taken care of.
The most interesting feature occurs when the two facilities do not agree on some aspect of care: the veteran does not need to be directly involved, resolution can be negotiated with minimal physician or staff input, and common cases can even be formalized and automated.
